% Team Note Sample Template
% These codes should be guaranteed, fast enough, short and easy to type.

\documentclass[landscape, 10pt, a4paper, oneside, twocolumn]{extarticle}
\usepackage{amssymb}
\usepackage{amsmath}
\usepackage{import}

\usepackage{teamnote}

\teamnote{PetrSU}{QA}{(Remeslennikov, Evstafeev, Titov)}

\ShowUsage
\ShowComplexity
\HideAuthor

\begin{document}

\maketitlepage

% Make Pagebreak if you want.
% \pagebreak 


\section{Graph}

\Algorithm
{Dinic}
{Almost linear in practice.}
{$\mathcal{O}(n^{2}m)$}
{cpp}{source/Dinic.cpp}

% \Algorithm
% {General Matching}
% {Use \texttt{init} to init, \texttt{addEdge} to add edges, \texttt{match} to get matching, \texttt{Match} to find maximum matching. Vertices have 1-based index.}
% {$\mathcal{O}(VE)$}
% {cpp}{source/GeneralMatching.cpp}

\section{Data Structure}

% \Algorithm
% {Randomized Meldable Heap}
% {Min-heap \texttt{H} is declared as \texttt{Heap<T> H}. You can use \texttt{push}, \texttt{size}, \texttt{empty}, \texttt{top}, \texttt{pop} as \texttt{std::priority\_queue}. Use \texttt{H.meld(G)} to meld contents from \texttt{G} to \texttt{H}. }
% {$\mathcal{O}(log n)$}
% {cpp}{source/MeldableHeap.cpp}

% \section{Geometry}

% \Algorithm
% {Smallest Enclosing Circle}
% {Use \texttt{solve} with \texttt{vector<Point>}. It returns \texttt{Circle c}, \texttt{c.p} is center, \texttt{c.r} is radius.}
% {$\mathcal{O}(n)$}
% {cpp}{source/SmallestEnclosingCircle.cpp}


\end{document}




